%%
%% 研究報告用スイッチ
%% [techrep]
%%
%% 欧文表記無しのスイッチ(etitle,jkeyword,eabstract,ekeywordは任意)
%% [noauthor]
%%

\documentclass[submit,techreq,noauthor]{ipsj}


\usepackage[dvips]{graphicx}
\usepackage{latexsym}

\def\Underline{\setbox0\hbox\bgroup\let\\\endUnderline}
\def\endUnderline{\vphantom{y}\egroup\smash{\underline{\box0}}\\}
\def\|{\verb|}

\setcounter{page}{1}

\pagestyle{empty}
\begin{document}


\title{Applied Type System}

\author{Hongwei Xi}{}{}

\begin{abstract}
Pure Type Systemフレームワーク ({\it PTS}) 型システムをデザイン/形式化する単純で一般的なアプローチを提供します。けれども依存型の存在を認めると、一般帰納、再帰型、作用 (例: 例外、参照、入出力)、などのような多くの実際のプログラミングの機能に {\it PTS} を適用させることが難しくなります。この論文では、実際のプログラミングの機能をサポートする型システムをたやすくデザイン/形式化できる、新しい Applied Type System ({\it ATS}) フレームワークを提案します。{\it ATS} の鍵となる突出した機能は、コンストラクトされて評価されるプログラムを含む動的な部分から、形作られて根拠となる型を含む静的な部分を完全に分離することにあります。この分離を用いると、{\it PTS} では許可されていましたが、プログラムが型の中に現われることは不可能になります。{\it ATS} の形式的な開発だけでなく、実用的なプログラミングのための型システムを作るフレームワークとして {\it ATS} を使ったいくつかの例も紹介します。

この翻訳の元論文は http://www.ats-lang.org/PAPER/ATS-types03.pdf です。
\end{abstract}

\maketitle
\thispagestyle{empty}

\section{はじめに}


\begin{acknowledgment}
\end{acknowledgment}

\begin{thebibliography}{10}
\end{thebibliography}

\end{document}
